% 9.5.07
% This is a sample documentation for Compass in the tex format.
% We restrict the use of tex to the following subset of commands:
%
% \section, \subsection, \subsubsection, \paragraph
% \begin{enumerate} (no-nesting), \begin{quote}, \item
% {\tt ... }, {\bf ...}, {\it ... }
% \htmladdnormallink{}{}
% \begin{verbatim}...\end{verbatim} is reserved for code segments
% ...''
%

\section{Non Associative Relational Operators}
\label{NonAssociativeRelationalOperators::overview}
{\it C++ Secure Coding Practices} states that:
\begin{quote}
The relational and equality operators are left-associative, not non-associative as they often are in other languages. This allows a C++ programmer to write an expression (particularly an expression used as a condition) that can be easily misinterpreted.
\end{quote}
This checker checks that relational binary operators $(==,!=,<,>,<=,>=)$ are not treated as if they were non-associative.

\subsection{Parameter Requirements}
This checker takes no parameters and inputs source file

\subsection{Implementation}
This pattern is checked using a nested AST traversal on the parent nodes of the operand of a binary operator expression. Any such parent node that treats relational binary operators as non-associative will use more than one binary relational operator. Flag these expressions as violations.

\subsection{Non-Compliant Code Example}

\begin{verbatim}
#include <stdio.h>

int main()
{
  int a = 2;
  int b = 2;
  int c = 2;

  if ( a < b < c ) // condition #1, misleading, likely bug
    printf( "a < b < c\n" );
  if ( a == b == c ) // condition #2, misleading, likely bug
    printf( "a == b == c\n" );

  return 0;
}
\end{verbatim}

\subsection{Compliant Solution}

\begin{verbatim}
#include <stdio.h>

int main()
{
  int a = 2;
  int b = 2;
  int c = 2;

  if ( a < b && b < c ) // clearer, and probably what was intended
    printf( "a < b && b < c\n" );
  if ( a == b && a == c ) // ditto
    printf( "a == b && a == c\n" );

  return 0;
}
\end{verbatim}

\subsection{Mitigation Strategies}
\subsubsection{Static Analysis} 

Compliance with this rule can be checked using structural static analysis checkers using the following algorithm:

\begin{enumerate}
\item Perform simple AST traversal and locate SgBinaryOp nodes.
\item At each SgBinaryOp node perform a nested traversal of the operands parent node and count the number of relational binary operators used.
\item If said count is greater than one then flag error.
\item Report all errors. 
\end{enumerate}

\subsection{References}

% Write some references
\htmladdnormallink{EXP09-A. Treat relational and equality operators as if they were nonassociative}{https://www.securecoding.cert.org/confluence/display/cplusplus/EXP09-A.+Treat+relational+and+equality+operators+as+if+they+were+nonassociative}
% ex. \htmladdnormallink{ISO/IEC 9899-1999:TC2}{https://www.securecoding.cert.org/confluence/display/seccode/AA.+C+References} Forward, Section 6.9.1, Function definitions''
