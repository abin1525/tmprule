% 9.5.07
% This is a sample documentation for Compass in the tex format.
% We restrict the use of tex to the following subset of commands:
%
% \section, \subsection, \subsubsection, \paragraph
% \begin{enumerate} (no-nesting), \begin{quote}, \item
% {\tt ... }, {\bf ...}, {\it ... }
% \htmladdnormallink{}{}
% \begin{verbatim}...\end{verbatim} is reserved for code segments
% ...''
%

\section{Data Member Access}
\label{DataMemberAccess::overview}

Following the spirit of data hiding in object-oriented programming, classes
should in general not have public data members as these might give away
details of the underlying implementation, making a change of implementation
more difficult and possibly giving users a way to mess up the internal state
of objects of the class. It is, however, sometimes useful to have
`behaviorless aggregates', i.\,e.~C-style structs where all data members are
public (and no member functions are present).

This checker warns about class definitions that fit into neither of the above
patterns. Specifically, it warns about class definitions that contain both
public and nonpublic data members.

\subsection{Parameter Requirements}

This checker does not require any parameters.

\subsection{Non-Compliant Code Example}

% write your non-compliant code subsection

\begin{verbatim}
class NoNo { // not acceptable, contains both public and nonpublic data members
public:
    int get_a() const;
    void set_a(int);

    double d;

protected:
    int a;
};
\end{verbatim}

\subsection{Compliant Solution}

% write your compliant code subsection

\begin{verbatim}
struct C_style { // acceptable, all data members are public
    int a, b;
    double d;
};

class WellProtected { // acceptable, no data members are public
public:
    int get_a() const;
    void set_a(int);

    double get_d() const;
    void set_d(double);

protected:
    int a;

private:
    double d;
};
\end{verbatim}

\subsection{Mitigation Strategies}
\subsubsection{Static Analysis} 

Compliance with this rule can be checked using structural static analysis checkers using the following algorithm:

\begin{enumerate}
\item For each class definition, count the numbers of public and nonpublic
data members.
\item If both of the counts for a given class definition are greater than
zero, a mix of public and nonpublic data members is present; emit a
diagnostic.
\end{enumerate}

\subsection{References}

% Write some references
% ex. \htmladdnormallink{ISO/IEC 9899-1999:TC2}{https://www.securecoding.cert.org/confluence/display/seccode/AA.+C+References} Forward, Section 6.9.1, Function definitions''
A literature reference for this checker is: H.~Sutter, A.~Alexandrescu: ``C++
Coding Standards"", Item 41: ``Make data members private, except in
behaviorless aggregates (C-style structs)''. Note that the authors advise
not only against public, but also against protected data members; this checker
does not report protected data.
