% 9.5.07
% This is a sample documentation for Compass in the tex format.
% We restrict the use of tex to the following subset of commands:
%
% \section, \subsection, \subsubsection, \paragraph
% \begin{enumerate} (no-nesting), \begin{quote}, \item
% {\tt ... }, {\bf ...}, {\it ... }
% \htmladdnormallink{}{}
% \begin{verbatim}...\end{verbatim} is reserved for code segments
% ...''
%

\section{Place Constant On The Lhs}
\label{PlaceConstantOnTheLhs::overview}
	 
This checker detects a test clause whether or not it contains a constant on the left hand side when comparing a varialbe and the constant for equality. By putting the constant on the left hand side, the compiler can prevent programmers from making mistake to write '=' for '=='. 

\subsection{Parameter Requirements}

%Write the Parameter specification here.
None

\subsection{Implementation}

%Details of the implementation go here.
   This checker is implemented with a simple traversal. It traverses AST and finds a test clause. If the test clause has a variable on its left hand side, then, then the checker report this clause to the standard output.

\subsection{Non-Compliant Code Example}

% write your non-compliant code subsection

\begin{verbatim}
void foo()
{
	int a = 0;

	if(a == 10) // a is on the LHS
	{
		a = 1;
	}
	
	while(a == 10) // a is on the LHS
	{
		a++;
	}

	do
	{
		a++;
	}while(a == 12); // a is on the LHS

	for(int i = 0; i == 0; i++) // i is on the LHS
	{
		a = 12;
	}
}
\end{verbatim}

\subsection{Compliant Solution}

% write your compliant code subsection

\begin{verbatim}
void foo()
{
	int a = 0;

	if(1 == a) // fine
	{
		a = 2;
	}
}
\end{verbatim}

\subsection{Mitigation Strategies}
\subsubsection{Static Analysis} 

Compliance with this rule can be checked using structural static analysis checkers using the following algorithm:

\begin{enumerate}
\item Check if the node visiting is an if statement.
\item If yes, find its test clause to see if it contains a constant on the left hand side.
\item Check also if the if statement compares a variable and the constant for equality
\end{enumerate}

\subsection{References}

% Write some references
% ex. \htmladdnormallink{ISO/IEC 9899-1999:TC2}{https://www.securecoding.cert.org/confluence/display/seccode/AA.+C+References} Forward, Section 6.9.1, Function definitions''
The Programming Research Group: High-Integrity C++ Coding Standard Manual, Item 10.6: ``When comparing variables and constants for equality always place place the constnat on the left hand side."
