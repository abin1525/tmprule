% 9.5.07
% This is a sample documentation for Compass in the tex format.
% We restrict the use of tex to the following subset of commands:
%
% \section, \subsection, \subsubsection, \paragraph
% \begin{enumerate} (no-nesting), \begin{quote}, \item
% {\tt ... }, {\bf ...}, {\it ... }
% \htmladdnormallink{}{}
% \begin{verbatim}...\end{verbatim} is reserved for code segments
% ...''
%

\section{Omp Private Lock}
\label{OmpPrivateLock::overview}
This is a simple checker to detect a common OpenMP programming mistake of
using a \lstinline{private} lock within a parallel region.
It was motivated by an example mentioned by a seminar about Intel's
Thread Checker, which uses a more expensive runtime approach to catch the
same error.

\subsection{Parameter Requirements}

%Write the Parameter specification here.
There is no parameter specifications.

\subsection{Implementation}

%Details of the implementation go here.
The AST must use  dedicated OpenMP nodes
(such as SgOmpParallelStatement etc.) to represent an input OpenMP program
(via \textit{-rose:openmp -rose:openmp:ast\_only}).
The checker traverses the AST to find all OpenMP lock variable references
(with a type name \lstinline{omp_lock_t})
within \lstinline{omp_unset_lock()}, \lstinline{omp_set_lock()}, and
\lstinline{omp_test_lock()}.
It then compares the scope of the corresponding lock declaration statement
and the enclosing parallel region of the lock references. 
If the lock is declared within the parallel region, it is a violation. 

\subsection{Non-Compliant Code Example}

% write your non-compliant code subsection

\begin{verbatim}

void foo()
{
#pragma omp parallel private(id)
  {
    omp_lock_t lck; // local declared lock variable, wrong!
    int id = omp_get_thread_num();
    omp_set_lock(&lck);
    printf("My thread id is %d.\n", id);
    omp_unset_lock(&lck);
  }
}

\end{verbatim}

\subsection{Compliant Solution}

% write your compliant code subsection

\begin{verbatim}

omp_lock_t lck; // a global shared lock, 
//lock initialization and destroy calls are omitted

void foo()
{
#pragma omp parallel private(id)
  {
    int id = omp_get_thread_num();
    omp_set_lock(&lck);
    printf("My thread id is %d.\n", id);
    omp_unset_lock(&lck);
  }
}
\end{verbatim}

%\subsection{Mitigation Strategies}
%\subsubsection{Static Analysis} 
%
%Compliance with this rule can be checked using structural static analysis checkers using the following algorithm:
%
%\begin{enumerate}
%\item Write your checker algorithm
%\end{enumerate}

\subsection{References}
None

% Write some references
% ex. \htmladdnormallink{ISO/IEC 9899-1999:TC2}{https://www.securecoding.cert.org/confluence/display/seccode/AA.+C+References} Forward, Section 6.9.1, Function definitions''
